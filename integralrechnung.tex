\chapter{Integralrechnung}

\section{Stammfunktionen}

\begin{flushleft}
    Stammfunktionen sind Funktionen, die eine Funktion $f$ ergeben, wenn man sie ableitet.
    Also gilt: $F'(x)=f(x)$.

    Allgemein gilt die folgende Regel.
    \begin{align}
        f(x)&=x^n \\
        F(x)&=\frac{x^{n+1}}{n+1}+C, n \in \mathbf{R} - \{-1\}
    \end{align}
    Wenn eine Funktion abgeleitet wird fällt jede Konstante weg, daher gibt es unendlich viele Stammfunktionen, die Konstante wird mit $C$ dargestellt.
\end{flushleft}

\section{Bestimmte Integrale}

\begin{flushleft}
    Bei bestimmten Integralen ist immer klar, welches Intervall gesucht ist.
    Also gilt für ein Integral in dem Intervall $[a;b]$ diese Formel:
    \begin{align}
        \int_{a}^{b} f(x) \ dx = [F(x)]_{a}^{b}
    \end{align}
    Außerdem ist die Integralrechnung keine Flächenberechnung, daher werden oft Absolutbeträge genutzt. Damit wird das Ergebnis eines Integrals immer positiv.
    Die Definition des Absolutbetrags sieht so aus:
    \[
        \mid x \mid =
        \begin{cases}
            x, &\text{wenn } x \geq 0 \\
            -x, &\text{sonst}
        \end{cases}
    \]
\end{flushleft}

\section{Flächen zwischen zwei Funktionen}

\begin{flushleft}
    Um die Fläche zwischen den beiden Funktionen $f$ und $g$ ($f \geq g$) im Intervall $[a;b]$ zu berechnen nutzt man diese Formel:
    \begin{align}
        \int_{a}^{b} [f(x)-g(x)] \ dx
    \end{align}
\end{flushleft}

\section{Mittelwerte von Funktionen}

\begin{flushleft}
    \textbf{diskrete Mittelwerte}: \newline
    Es wird ein Mittelwert aus einer Menge von Zahlen gebildet. \newline
    \begin{align}
        \{1,3&,5,4\} \\
        \frac{1+3+5+4}{4}&=\frac{13}{4}=3.25
    \end{align}
    \newline
    \textbf{kontinuierliche Mittelwerte}: \newline
    Der Mittelwert wird aus Werten einer Funktion bestimmt. \newline
    \begin{align}
        \bar{m}=\frac{1}{b-a}\int_{a}^{b} f(x) \ dx
    \end{align}
\end{flushleft}

\section{Rekonstruktion von Beständen}

\begin{flushleft}
    Bei diesem Aufgabentyp hat man immer eine Funktion (hier: $f$) gegeben, die die momentane Änderungsrate beschreiben soll, also wie stark etwas ansteigt oder fällt. Um alle Teilaufgaben vernünftig zu lösen muss man erstmal die Stammfunktion bestimmen:
    \begin{align}
        f(x)&=F'(x) \\
        F(x)&=\int f(x) \ dx
    \end{align}
    Die Konstante $C$, die beim integrieren erscheint sollte in der Aufgabenstellung gegeben sein.
\end{flushleft}
