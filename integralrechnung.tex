\chapter{Integralrechnung}

\section{Stammfunktionen}

\begin{flushleft}
    Stammfunktionen sind Funktionen, die eine Funktion $f$ ergeben, wenn man sie ableitet.
    Also gilt: $F'(x)=f(x)$.

    Allgemein gilt die folgende Regel.
    \begin{align}
        f(x)&=x^n \\
        F(x)&=\frac{x^{n+1}}{n+1}+C, n \in \mathbf{R} - \{-1\}
    \end{align}
    Wenn eine Funktion abgeleitet wird fällt jede Konstante weg, daher gibt es unendlich viele Stammfunktionen, die Konstante wird mit $C$ dargestellt.
\end{flushleft}

\section{Bestimmte Integrale}

\begin{flushleft}
    Bei bestimmten Integralen ist immer klar, welcher Intervall gesucht ist.
    Also gilt für ein Integral in dem Intervall $[a;b]$ die folgende Formel.
    \begin{align}
        \int_{a}^{b} f(x) \ dx = [F(x)]_{a}^{b}
    \end{align}
    Außerdem ist die Integralrechnung keine Flächenberechnung, daher werden oft Absolutbeträge genutzt. Damit wird das Ergebnis eines Integrals immer positiv.
    Die Definition des Absolutbetrags sieht aus wie folgt.
    \[
        \mid x \mid =
        \begin{cases}
            x, &\text{wenn } x \geq 0 \\
            -x, &\text{sonst}
        \end{cases}
    \]
\end{flushleft}

\section{Flächen zwischen zwei Funktionen}

\begin{flushleft}
    Um die Fläche zwischen zwei Funktionen $f$ und $g$ ($f \geq g$) im Intervall $[a;b]$ zu berechnen nutzt man meist die folgende Formel.
    \begin{align}
        \int_{a}^{b} [f(x)-g(x)] \ dx
    \end{align}
\end{flushleft}
