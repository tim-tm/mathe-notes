\chapter{Analysis}

\section{Funktionsscharen}
\subsection{Definition}
\begin{flushleft}
    Eine Funktionsschar ist eine Funktion, die auch von anderen Parametern als $x$ abhängt.
    Durch einen weiteren Parameter kann es plötzlich unendlich viele Variationen von der ursprünglichen Funktion geben.
    Das ist auch der Grund dafür, dass solch eine Funktion, Funktionsschar genannt wird.
    Es entsteht eine Menge -- also ein Schar -- von Funktionen.
\end{flushleft}

\subsection{Beispiel}
\begin{flushleft}
    Das ist beispielsweise eine quadratische Funktionsschar, die von den Parametern $x$ und $k$ abhängt.
    \begin{align}
        f_k(x)=kx^2
    \end{align}
    Der Parameter $k$ bestimmt nur die Stauchung der Funktion.
\end{flushleft}

\begin{center}
\begin{tikzpicture}
\begin{axis}[
    axis lines=middle,
    samples=35,
    domain=-5:5,
    every axis plot/.append style={thick},
]
\addplot[
    color=red,
]
{x^2};
\addlegendentry{$x^2$}

\addplot[
    color=green,
]
{0.5*x^2};
\addlegendentry{$\frac{1}{2}x^2$}

\addplot[
    color=blue,
]
{0.3333*x^2};
\addlegendentry{$\frac{1}{3}x^2$}

\addplot[
    color=orange,
]
{0.25*x^2};
\addlegendentry{$\frac{1}{4}x^2$}

\end{axis}
\end{tikzpicture}
\end{center}

\section{Trassierungsaufgaben}
\subsection{Definition}
\begin{flushleft}
    Das Ziel von Trassierungsaufgaben ist es anhand von bestimmten Angaben eine Funktion aufzustellen.
    Deshalb ist das Verstehen der Aufgabenstellung der schwierigste Teil der Aufgabe.
\end{flushleft}

\subsection{Beispiel}
\begin{flushleft}
    Das ist eine mögliche Aufgabenstellung: \\
    \textit{Eine quadratische Funktion hat im Punkt $P(0|0)$ eine Nullstelle und im Punkt $Q(2|5)$ ein Maximum. Stelle die Funktionsgleichung auf!} \\
    \begin{enumerate}
        \item {
                Die ersten drei Wörter des Satzes: \textbf{\textit{Eine quadratische Funktion}} sagen uns um welchen Typ Funktion es sich handelt.
                Es ist eine quadratische Funktion. Die allgemeine quadratische Funktion sieht so aus:
                \begin{align}
                    f(x)=ax^2+bx+c
                \end{align}
                Unsere Funktion $f$ hat also drei unbekannte, $a$, $b$ und $c$.
            }
        \item {
                Außerdem ist für uns wichtig, dass die Funktion \textbf{\textit{im Punkt $P(0|0)$ eine Nullstelle}} hat.
                Das bedeutet, dass $f(0)=0$ sein muss.
                Unsere erste Bedingung ist also:
                \begin{align}
                    f(0)=0
                \end{align}
            }
        \item {
                Als letztes hat unsere Funktion \textbf{\textit{im Punkt $Q(2|5)$ ein Maximum}}. \\
                Um die Extrempunkte von einer Funktion (hier: $g$) zu finden, muss man die erste Ableitung dieser Funktion gleich $0$ setzen.
                \begin{align}
                    g'(x)=0
                \end{align}
                Im Idealfall bekommt man eine oder mehr Lösungen heraus, wir nennen die Lösung $x_1$.
                Um herauszufinden um welche Art von Extrempunkt es sich handelt setzen wir $x_1$ in die zweite Ableitung von $g$ ein.
                \begin{align}            
                    g''(x) =
                    \begin{cases}
                        \text{Maximum}, &\text{wenn } x < 0 \\
                        \text{Minimum}, &\text{wenn } x > 0
                    \end{cases}
                \end{align}
                Jetzt können wir drei weitere Bedingungen für unsere Aufgabe aufstellen.
                Unsere Funktion soll ein Maximum in $Q(2|5)$ haben, deshalb müssen diese Bedingungen erfüllt sein:
                \begin{align}
                    f(2)=5 \\
                    f'(2)=0 \\
                    f''(2) < 0
                \end{align}
            }
        \item {
                Jetzt müssen wir aus unseren vier Bedingungen eine Funktion bilden, dafür bilden wir erstmal die ersten beiden Ableitungen.
                \begin{align}
                    f(x)&=ax^2+bx+c \\
                    f'(x)&=2ax+b \\
                    f''(x)&=2a \\
                    f(0)&=0 \\
                    f(2)&=5 \\
                    f'(2)&=0 \\
                    f''(2) & < 0
                \end{align}
                Nun müssen wir einsetzen, Ungleichungen ($f''(2) < 0$) dürfen nicht eingesetzt werden.
                \begin{align}
                    0^2a+0b+c &=0 \\
                    c &=0 \\
                    a*2^2+2b+c &= 5 \\
                    4a+2b+c &= 5 \\
                    4a+2b &= 5 \\
                    2a*2+b &=0 \\
                    4a+b &= 0
                \end{align}
                Eine von den drei unbekannten ist gelöst ($c=0$).
                Jetzt muss $4a+2b=5$ in $4a+b=0$ eingesetzt werden.
                \begin{align}
                    4a+2b &=5 \\
                    4a+b &= 0 \\
                    4a+2b-(4a+b) &= 5 \\
                    b &= 5 \\
                    4a+5 &= 0 \\
                    a &= \frac{-5}{4}
                \end{align}
                Alle unbekannten Variablen sind gelöst, $a=\frac{-5}{4}$, $b=5$, $c=0$.
                Also ist das unsere Funktion:
                \begin{align}
                    f(x)=\frac{-5}{4}x^2+5x
                \end{align}
            }
    \end{enumerate}
\end{flushleft}

\begin{center}
\begin{tikzpicture}
\begin{axis}[
    axis lines=middle,
    samples=35,
    domain=-1:5,
    ymax=5.5,
    every axis plot/.append style={thick},
]
\addplot[
    color=red,
]
{(-5/4)*((\x)*(\x))+(5)*(\x)};
\addlegendentry{$f(x)$}

\end{axis}
\end{tikzpicture}
\end{center}

\section{Extremwertaufgaben}
\subsection{Definition}
\begin{flushleft}   
    Bei Extemwertaufgaben geht es darum die maximal/minimal möglichen Flächeninhalte herauszufinden.
    Dabei kann es oft auch um das maximale/minimale Volumen eines Körpers gehen.
    Jedoch maximiert/minimiert sich das Volumen eines Körpers wenn der Flächeninhalt des Mantels maximal/minimal wird.
    Die Aufgabenstellungen zu diesem Aufgabentyp sind meistens sehr spezifisch und brauchen eine Skizze.
\end{flushleft}

\subsection{Beispiel}
\begin{flushleft}
    In diesem Beispiel soll der maximale Flächeninhalt eines Dreiecks gefunden werden.
    Das Dreieck ist unterhalb der Funktion $f$.
    Die drei Ecken des Dreiecks sind die Punkte $P(u|v)$, $Q(0|0)$ und $R(u|0)$.
    \begin{align}
        f(x)=x^3-6x^2+9x
    \end{align}
\end{flushleft}

\begin{center}
\begin{tikzpicture}   
\begin{axis}[
    axis lines=middle,
    samples=35,
    domain=-1.5:5,
    ymax=7,
    ymin=-5,
    every axis plot/.append style={thick},
]
\addplot[
    color=red,
    name path=A
]
{((\x)*(\x)*(\x))-(6*(\x)*(\x))+(9*(\x))};
\addlegendentry{$f(x)$}

\addplot[draw=none,name path=B] {(3.4/1.5)*(\x)};
\addplot[draw=none,name path=C] {0};
\addplot[red,opacity=0.3] fill between[of=B and C,soft clip={domain=0:1.5}];

\node[label={150:{$Q(0|0)$}},circle,fill,inner sep=1pt] at (axis cs:0,0) {};
\node[label={70:{$P(u|v)$}},circle,fill,inner sep=1pt] at (axis cs:1.5,3.4) {};
\node[label={60:{$R(u|0)$}},circle,fill,inner sep=1pt] at (axis cs:1.5,0) {};

\end{axis}
\end{tikzpicture}
\end{center}

\begin{flushleft}
    Um die Fläche $A$ eines Dreiecks auszurechnen nutzt man diese Formel, $g$ steht für die Grundseite und $h$ für die Höhe:
    \begin{align}
        A=\frac{gh}{2}
    \end{align}
    Mit $g$ und $h$ sieht der Plot so aus:
\end{flushleft}

\begin{center}
\begin{tikzpicture}   
\begin{axis}[
    axis lines=middle,
    samples=35,
    domain=-1.5:4,
    ymax=7,
    ymin=-5,
    every axis plot/.append style={thick},
]
\addplot[
    color=red,
    name path=A
]
{((\x)*(\x)*(\x))-(6*(\x)*(\x))+(9*(\x))};
\addlegendentry{$f(x)$}

\addplot[draw=none,name path=B] {(3.4/1.5)*(\x)};
\addplot[draw=none,name path=C] {0};
\addplot[red,opacity=0.3] fill between[of=B and C,soft clip={domain=0:1.5}];

\node[label={150:{$Q(0|0)$}},circle,fill,inner sep=1pt] at (axis cs:0,0) {};
\node[label={70:{$P(u|v)$}},circle,fill,inner sep=1pt] at (axis cs:1.5,3.4) {};
\node[label={270:{$R(u|0)$}},circle,fill,inner sep=1pt] at (axis cs:1.5,0) {};

\node[label={0:{$h$}},draw=none,inner sep=1pt] at (axis cs:1.5,1.5) {};
\node[label={250:{$g$}},draw=none,inner sep=1pt] at (axis cs:0.75,0) {};

\end{axis}
\end{tikzpicture}
\end{center}

\begin{flushleft}
    In unserem Beispiel ist $g=u$ und $h=v$, dazu ist $v=f(u)$.
    Wenn wir diese Werte in $A$ einsetzen bildet sich diese Funktion:
    \begin{align}
        A(u) &=\frac{u*f(u)}{2} \\
        A(u) &=\frac{u \left(u^3-6u^2+9u\right)}{2} \\
        A(u) &=\frac{u^4-6u^3+9u^2}{2} \\
        A(u) &=\frac{u^4}{2}-3u^3+\frac{9}{2}u^3
    \end{align}
    Unsere neue Funktion $A(u)$ kann jetzt den Flächeninhalt für jeden Wert von $u$ berechnen.
    Wir brauchen den maximalen Flächeninhalt, also soll die Funktion für den Flächeninhalt ($A(u)$) maximal werden.
    Um ein maximum zu berechnen bilden wir die ersten zwei Ableitungen unserer Funktion.
    \begin{align}
        A(u) &=\frac{u^4}{2}-3u^3+\frac{9}{2}u^3 \\
        A'(u) &= 2u^3-9u^2+9u \\
        A''(u) &= 6u^2-18u+9
    \end{align}
    Jetzt finden wir die Nullstelle der ersten Ableitung.
    \begin{align}
        A'(u) &=0 \\
        \iff u_1=0, u_2 &=\frac{3}{2}, u_3=3
    \end{align}
    Danach prüfen wir, in welcher Stelle ein Maximum liegt.
    \begin{align}
        A''\left(0\right) = 9 & > 0 \Longrightarrow \text{Minimum} \\
        A''\left(\frac{3}{2}\right) = \frac{-9}{2} & < 0 \Longrightarrow \text{Maximum} \\
        A''\left(3\right) = 9 & > 0 \Longrightarrow \text{Minimum} \\
        A\left(\frac{3}{2}\right) &= \frac{81}{32} \\
        v=f\left(\frac{3}{2}\right) &= \frac{27}{8}
    \end{align}
    Der größte Flächeninhalt ist $\frac{81}{32}$, um diesen zu erreichen muss $u=\frac{3}{2}$ und $v=\frac{27}{8}$ sein.
\end{flushleft}

\begin{center}
\begin{tikzpicture}   
\begin{axis}[
    axis lines=middle,
    samples=35,
    domain=-1.5:4,
    ymax=7,
    ymin=-5,
    every axis plot/.append style={thick},
]
\addplot[
    color=red,
    name path=A
]
{((\x)*(\x)*(\x))-(6*(\x)*(\x))+(9*(\x))};
\addlegendentry{$f(x)$}

\addplot[draw=none,name path=B] {((27/8)/(3/2))*(\x)};
\addplot[draw=none,name path=C] {0};
\addplot[red,opacity=0.3] fill between[of=B and C,soft clip={domain=0:1.5}];

\addplot[color=black,very thick] coordinates {(1.5,0) (1.5,3.375)};
\addplot[color=black,very thick] coordinates {(0,0) (1.5,0)};

\node[label={0:{$\frac{27}{8}$}},draw=none,inner sep=1pt] at (axis cs:1.5,1.5) {};
\node[label={250:{$\frac{3}{2}$}},draw=none,inner sep=1pt] at (axis cs:0.75,0) {};

\end{axis}
\end{tikzpicture}
\end{center}

\section{Integralrechnung}
\subsection{Stammfunktionen}
\begin{flushleft}
    Stammfunktionen sind Funktionen, die eine Funktion $f$ ergeben, wenn man sie ableitet.
    Also gilt: $F'(x)=f(x)$.

    Allgemein gilt die folgende Regel.
    \begin{align}
        f(x)&=x^n \\
        F(x)&=\frac{x^{n+1}}{n+1}+C, n \in \mathbf{R} - \{-1\}
    \end{align}
    Wenn eine Funktion abgeleitet wird fällt jede Konstante weg, daher gibt es unendlich viele Stammfunktionen, die Konstante wird mit $C$ dargestellt.
\end{flushleft}

\subsection{Bestimmte Integrale}
\begin{flushleft}
    Bei bestimmten Integralen ist immer klar, welches Intervall gesucht ist.
    Also gilt für ein Integral in dem Intervall $[a;b]$ diese Formel:
    \begin{align}
        \int_{a}^{b} f(x) \ dx = [F(x)]_{a}^{b}
    \end{align}
    Außerdem ist die Integralrechnung keine Flächenberechnung, daher werden oft Absolutbeträge genutzt. Damit wird das Ergebnis eines Integrals immer positiv.
    Die Definition des Absolutbetrags sieht so aus:
    \[
        \left | x \right | =
        \begin{cases}
            x, &\text{wenn } x \geq 0 \\
            -x, &\text{sonst}
        \end{cases}
    \]
    Dass die Integralrechnung keine Flächenberechnung ist kann man sich schnell deutlich machen.
    Haben wir beispielsweise die Funktion $f(x)=-x^2+4$ gegeben und wollen diese von $-2$ bis $2$ integrieren, sieht die Fläche, die wir berechnen wollen so aus:
\end{flushleft}

\begin{center}
\begin{tikzpicture}
\begin{axis}[
        xmin=-3,
        xmax=3,
        ymax=5,
        ymin=-2,
        domain=-3:3,
        samples=35,
        axis lines=middle,
        %unbound coords=discard,
    ]
    \addplot[red,thick,name path=A] ({\x},{-((\x)*(\x))+4});
    \addlegendentry{\(f(x)=-x^2+4\)}
    
    \addplot[draw=none,name path=B] {0};
    \addplot[red,opacity=0.3] fill between[of=A and B,soft clip={domain=-2:2}];
\end{axis}
\end{tikzpicture}
\end{center}

\begin{flushleft}
    Jetzt bilden wir das Integral und lösen es auf.
    \begin{align}
        &\int_{-2}^{2} -x^2+4 \ dx \\
        = &\left[\frac{-x^3}{3}+4x\right]_{-2}^{2} \\
        = &\left[\frac{-2^3}{3}+4*2-\left(\frac{-(-2)^3}{3}+4(-2)\right)\right] \\
        = &\frac{32}{3}
    \end{align}
    Es gibt keine Probleme. Möchten wir jetzt jedoch von $2$ bis $4$ integrieren, stellen wir etwas komisches fest.
\end{flushleft}

\begin{center}
\begin{tikzpicture}
\begin{axis}[
        xmin=0,
        xmax=6,
        ymax=5,
        ymin=-6,
        domain=0:6,
        samples=35,
        axis lines=middle,
        %unbound coords=discard,
    ]
    \addplot[red,thick,name path=A] ({\x},{-((\x)*(\x))+4});
    \addlegendentry{\(f(x)=-x^2+4\)}
    
    \addplot[draw=none,name path=B] {0};
    \addplot[red,opacity=0.3] fill between[of=A and B,soft clip={domain=2:4}];
\end{axis}
\end{tikzpicture}
\end{center}

\begin{flushleft}
    Erstmal das Integral bilden.
    \begin{align}
        &\int_{2}^{4} -x^2+4 \ dx \\
        = &\left[\frac{-x^3}{3}+4x\right]_{2}^{4} \\
        = &\left[\frac{-4^3}{3}+4*4-\left(\frac{-2^3}{3}+4*2\right)\right] \\
        = &\frac{-32}{3}
    \end{align}
    Das Ergebnis ist negativ, obwohl ein negativer Flächeninhalt nicht möglich ist.
    Deshalb nutzen wir, wenn das Ergebnis des Integrals negativ ist -- was vorher geprüft werden sollte -- den Betrag des Ergebnisses.
    \begin{align}
        & \left | \int_{2}^{4} -x^2+4 \ dx \right | \\
        = & \left | \frac{-32}{3} \right | \\
        = &\frac{32}{3}
    \end{align}
\end{flushleft}

\subsection{Flächen zwischen zwei Funktionen}
\begin{flushleft}
    Um die Fläche zwischen den beiden Funktionen $f$ und $g$ ($f \geq g$) im Intervall $[a;b]$ zu berechnen nutzt man diese Formel:
    \begin{align}
        \int_{a}^{b} \left[f(x)-g(x)\right] \ dx
    \end{align}
    Beispiel: $f(x)=-x^2+4$, $g(x)=2$
\end{flushleft}

\begin{center}
\begin{tikzpicture}
\begin{axis}[
        xmin=-6,
        xmax=6,
        ymax=5,
        ymin=-1,
        domain=-6:6,
        samples=35,
        axis lines=middle,
        %unbound coords=discard,
    ]
    \addplot[red,thick,name path=A] ({\x},{-((\x)*(\x))+4});
    \addlegendentry{\(f(x)\)}
    \addplot[blue,thick,name path=B] ({\x},{2});
    \addlegendentry{\(g(x)\)}
    
    \addplot[red,opacity=0.3] fill between[of=A and B,soft clip={domain=-1.5:1.5}];
\end{axis}
\end{tikzpicture}
\end{center}

\begin{flushleft}
    Um die Fläche zwischen $f$ und $g$ zu bestimmen muss zunächst klar werden, welche Funktion in dem gezeigten Bereich größer ist.
    Hier ist $f > g, [x_1;x_2]$, $x_1$ und $x_2$ stellen hier die Schnittpunkte der beiden Funktionen dar.
    Jetzt darf einfach in die Formel eingesetzt werden, $x_1$ und $x_2$ müssen dafür natürlich erst bestimmt werden, das werden wir in diesem Beispiel jedoch nicht machen.
    \begin{align}
        \int_{x_1}^{x_2} \left[f(x)-g(x)\right] \ dx
    \end{align}
\end{flushleft}

\subsection{Mittelwerte von Funktionen}
\begin{flushleft}
    \textbf{diskrete Mittelwerte}: \newline
    Es wird ein Mittelwert aus einer Menge von Zahlen gebildet. \newline
    \begin{align}
        \{1,3&,5,4\} \\
        \frac{1+3+5+4}{4}&=\frac{13}{4}=3.25
    \end{align}
    \newline
    \textbf{kontinuierliche Mittelwerte}: \newline
    Der Mittelwert wird aus Werten einer Funktion bestimmt. \newline
    \begin{align}
        \bar{m}=\frac{1}{b-a}\int_{a}^{b} f(x) \ dx
    \end{align}
    Beispiel: $f(x)=-x^2+4$
\end{flushleft}

\begin{center}
\begin{tikzpicture}
\begin{axis}[
        xmin=-4,
        xmax=4,
        ymax=5,
        ymin=-1,
        domain=-4:4,
        samples=35,
        axis lines=middle,
        %unbound coords=discard,
    ]
    \addplot[red,thick,name path=A] ({\x},{-((\x)*(\x))+4});
    \addlegendentry{\(f(x)\)}
\end{axis}
\end{tikzpicture}
\end{center}

\begin{flushleft}
    Wir wollen den Mittelwert $\bar{m}$ von $f$ in dem Interval $\left[-2;2\right]$ bestimmen.
    Dafür setzen wir einfach in die Formel ein.
    \begin{align}
        \bar{m}&=\frac{1}{b-a}\int_{a}^{b} f(x) \ dx \\
        \bar{m}&=\frac{1}{2-(-2)}\int_{-2}^{2} -x^2+4 \ dx \\
        \bar{m}&=\frac{1}{4}\int_{-2}^{2} -x^2+4 \ dx \\
        \bar{m}&=\frac{1}{4}\left[\frac{-x^3}{3}+4x\right]_{-2}^{2} \\
        \bar{m}&=\frac{1}{4}\left[\frac{-2^3}{3}+4*2-\left(\frac{-(-2)^3}{3}+4(-2)\right)\right] \\
        \bar{m}&=\frac{1}{4}*\frac{32}{3} \\
        \bar{m}&=\frac{8}{3}
    \end{align}
    Der Mittelwert der Funktion $f$ im Interval $\left[-2;2\right]$ ist also $\frac{8}{3}$.
\end{flushleft}

\begin{center}
\begin{tikzpicture}
\begin{axis}[
        xmin=-4,
        xmax=4,
        ymax=5,
        ymin=-1,
        domain=-4:4,
        samples=35,
        axis lines=middle,
        %unbound coords=discard,
    ]
    \addplot[red,thick,name path=A] ({\x},{-((\x)*(\x))+4});
    \addlegendentry{\(f(x)\)}
    
    \addplot[blue,thick,name path=B] ({\x},{8/3});
    \addlegendentry{\(\bar{m}\)}
\end{axis}
\end{tikzpicture}
\end{center}

\subsection{Rekonstruktion von Beständen}
\begin{flushleft}
    Bei diesem Aufgabentyp hat man immer eine Funktion (hier: $f$) gegeben, die die momentane Änderungsrate beschreiben soll, also wie stark etwas ansteigt oder fällt.
    Um alle Teilaufgaben vernünftig zu lösen muss man erstmal die Stammfunktion bestimmen:
    \begin{align}
        f(x)&=F'(x) \\
        F(x)&=\int f(x) \ dx
    \end{align}
    Die Konstante $C$, die beim integrieren erscheint sollte in der Aufgabenstellung gegeben sein.
\end{flushleft}

\subsection{Rotationskörper}
\begin{flushleft}
    Rotationskörper, sind Objekte, die sich bilden wenn eine Funktion um eine Achse rotiert.
    Die Rotationsachse wird auch die Figureachse genannt.
    Um allgemein das Volumen eines Rotationskörpers, der durch die Funktion $f$ beschrieben wird zu bestimmen nutzt man diese Formel:
    \begin{align}
        V=\pi\int_{a}^{b}\left[f(x)\right]^2 \ dx
    \end{align}
    Beispiel: $f(x)=x^2$, im Interval $[0;2]$
\end{flushleft}

\begin{center}
\begin{tikzpicture}
\begin{axis}[
        xmin=-1,
        xmax=4,
        ymax=5,
        ymin=-1,
        domain=-1:4,
        samples=35,
        axis lines=middle,
        %unbound coords=discard,
    ]
    \addplot[red,thick,name path=A] ({\x},{((\x)*(\x))});
    \addlegendentry{\(f(x)\)}
    
    \addplot[draw=none,name path=B] ({\x},{0});
    \addplot[red,opacity=0.3] fill between[of=A and B,soft clip={domain=0:2}];
\end{axis}
\end{tikzpicture}
\end{center}

\begin{flushleft}
    Wenn diese Funktion $f$ um die X-Achse rotiert wird, sieht der Körper, der entsteht in etwa so aus:
\end{flushleft}

% Code von: https://tex.stackexchange.com/questions/191091/function-rotated-about-the-x-axis
\begin{center}
    \def\aDomain{0}
    \def\bDomain{2}

    \def\fcn{\x^2}

    \def\cRange{0}
    \def\dRange{4}

    \def\xGridSteps{8}

    \def\rotationGridSteps{16}

    \def\phi{17}

\begin{tikzpicture}[domain= \aDomain: \bDomain]
    \pgfmathsetmacro\intervalLength{\bDomain - \aDomain}
    \pgfmathsetmacro\xGridStepsize{\intervalLength/\xGridSteps}
    \pgfmathsetmacro\rotationGridStepsize{360/\rotationGridSteps}

    \foreach \theta in {0, \rotationGridStepsize, ..., 180} {
        \tikzset{xyplane/.estyle={cm={
        cos(\phi), 0, sin(\theta)*sin(\phi),
        cos(\theta), (0, 0)}}}
        \draw[xyplane,smooth] plot (\x, \fcn) ;
    }

    \pgfmathsetmacro\nextStep{\aDomain + \xGridStepsize}
    \foreach \x in {\aDomain,\nextStep, ...,\bDomain} {
        \pgfmathsetmacro\xsh{(cos(\phi))*(\x)}
        \pgfmathsetmacro\rad{(\fcn)}        
        \tikzset{xyplane/.estyle={cm={cos(\phi - 90), 0,0,1,(\xsh, 0)}}}        
        \draw[xyplane,black,thin,opacity=1] (0, \rad) arc (90 : 270 : \rad);
    }

    \foreach \theta in {0, \rotationGridStepsize, ..., 180} {
        \tikzset{xyplane/.estyle={cm={cos(\phi), 0, sin(\theta)*sin(\phi),cos(\theta),(0, 0)}}}
        \draw[xyplane,smooth] plot (\x, \fcn) ;
    }

    \foreach \x in {\aDomain, \nextStep, ..., \bDomain}{
        \pgfmathsetmacro\xsh{(cos(\phi))*(\x)}
        \pgfmathsetmacro\rad{(\fcn)}
        \tikzset{xyplane/.estyle={cm={cos(\phi-90),0,0,1, (\xsh, 0)}}}
        \draw[xyplane] (0, -\rad) arc (-90 : 90 : \rad);
    }
\end{tikzpicture}
\end{center}

\begin{flushleft}
    Jetzt interessiert uns das Volumen $V$, das dieser Körper hat.
    Dazu setzen wir unsere Werte in die Formel ein.
    \begin{align}
        V &=\pi\int_{0}^{2}\left[x^2\right]^2 \ dx \\
        V &=\pi\int_{0}^{2} x^4 \ dx \\
        V &=\pi \left[\frac{x^5}{5}\right]_{0}^{2} \\
        V &=\pi \left[\frac{2^5}{5}-\left(\frac{0^5}{5}\right)\right] \\
        V &=\pi \frac{32}{5} \\
        V &=\frac{32\pi}{5}
    \end{align}
    Nun wissen wir, dass das Volumen $V$ unseres Rotationskörpers $\frac{32\pi}{5}$ ist.
\end{flushleft}

\subsection{Uneigentliche Integrale}
\begin{flushleft}
    Uneigentliche Integrale sind bestimmte Integrale mit komplizierteren Integrationsgrenzen.
    Oft wird $-\infty$ oder $+\infty$ in die Integrationsgrenzen eingebaut, es können jedoch auch Integrale mit richtigen Zahlen uneigentlich sein.
\end{flushleft}

\begin{center}
\begin{tikzpicture}
\begin{axis}[
        xmin=0,
        xmax=5,
        ymax=5,
        ymin=0,
        domain=0.05:5,
        samples=35,
        axis y line=left,
        axis x line=bottom,
        restrict y to domain=0:5
        %unbound coords=discard,
    ]
    \addplot[red,thick,name path=A] ({\x},{1/((\x)*(\x))});
    \addlegendentry{\(f(x)=\frac{1}{x^2}\)}
\end{axis}
\end{tikzpicture}
\end{center}

\begin{flushleft}
    Hier sieht man den Plot von $f(x)=\frac{1}{x^2}$.
    Es lässt sich relativ gut erkennen, dass der Wert der Funktion immer größer wird, umso näher man der Y-Achse kommt.
    Entfernt man sich also von der Y-Achse, wird der Wert immer kleiner.
    Mathematisch lässt sich das so ausdrücken:
    \begin{align}
        \lim_{x \to 0} f(x) &= \infty \\
        \lim_{x \to \infty} f(x) &= 0 \\
        \lim_{x \to -\infty} f(x) &= 0
    \end{align}
    Diese Grenzen muss man beachten, wenn man uneigentliche Integrale, wie beispielsweise dieses Integral ausrechnen möchte.
    \begin{align}
        &\int_{1}^{\infty} f(x) \ dx
    \end{align}
    Mit diesem Integral würde man die folgende Fläche berechnen:
\end{flushleft}

\begin{center}
\begin{tikzpicture}
\begin{axis}[
        xmin=0,
        xmax=5,
        ymax=5,
        ymin=0,
        domain=0.05:5,
        samples=35,
        axis y line=left,
        axis x line=bottom,
        restrict y to domain=0:5
        %unbound coords=discard,
    ]
    \addplot[red,thick,name path=A] ({\x},{1/((\x)*(\x))});
    \addlegendentry{\(f(x)=\frac{1}{x^2}\)}
    
    \addplot[draw=none,name path=B] {0};
    \addplot[red,opacity=0.3] fill between[of=A and B,soft clip={domain=1:5}];
\end{axis}
\end{tikzpicture}
\end{center}

\begin{flushleft}
    Wenn man versucht dieses Integral zu berechnen fällt eine Besonderheit auf.
    \begin{align}
        &\int_{1}^{\infty} f(x) \ dx \\
        &\int_{1}^{\infty} \frac{1}{x^2} \ dx \\
        &\int_{1}^{\infty} x^{-2} \ dx \\
        &\left[\frac{x^{-1}}{-1}\right]_{1}^{\infty} \\
        &\left[\frac{-1}{x}\right]_{1}^{\infty}
    \end{align}
    Als obere Grenze kann nicht einfach $\infty$ eingesetzt werden, deshalb suchen wir einen Weg um uns an $\infty$ anzunähern.
    \begin{align}
        &\lim_{b \to \infty} \left[ \frac{-1}{x}\right]_{1}^{b}
    \end{align}
    Jetzt ist es möglich anstatt $\infty$ einfach unsere Variable $b$ einzusetzen und erstmal so weit wie möglich aufzulösen.
    \begin{align}
        &\lim_{b \to \infty} \left[ \frac{-1}{b} - \left(\frac{-1}{1}\right)\right] \\
        &\lim_{b \to \infty} \left( \frac{-1}{b} + 1\right)
    \end{align}
    Nachdem wir so weit wie möglich vereinfacht haben, gucken wir uns jetzt jeden Bestandteil des Ergebnisses an.
    Da unsere Variable $b$ ist, interessieren uns erstmal nur die Terme, in denen $b$ vorkommt.
    $+1$ ist für den Limes irrelevant, da es kein $b$ enthält, $\frac{-1}{b}$ ist jedoch relevant, deshalb müssen wir uns $\frac{-1}{b}$ genauer angucken. \newline
    Dazu müssen wir uns fragen, was passiert, wenn $b$ einen sehr großen Wert annimmt.
    Anhand von ein paar Beispielen kann man das verhalten von $\frac{-1}{b}$ relativ gut verinnerlichen.
    Man kann zur Veranschaulichung einfach ein paar Werte für $b$ einsetzen.
    \begin{align}
        &\frac{-1}{b} \\ 
        \frac{-1}{2} &= -0.5 \\ 
        \frac{-1}{4} &= -0.25 \\ 
        \frac{-1}{8} &= -0.125 \\ 
        \frac{-1}{16} &= -0.0625
    \end{align}
    Wie man relativ gut erkennen kann, geht $\frac{-1}{b}$ immer weiter gegen $0$, desto größer der Wert für $b$ ist.
    Das bedeutet für unsere ursprüngliche Frage, dass dieser Term wegfällt.
    \begin{align}
        \lim_{b \to \infty} &\left( \frac{-1}{b} + 1\right) \\
        &=1 \\
        \int_{1}^{\infty} f(x) \ dx &= 1
    \end{align}
    Nach ein wenig Rechnerei wissen wir also, dass unser uneigentliches Integral, welches eine obere Grenze von $\infty$ hat, nicht den Wert von $\infty$, 
    sondern einen Wert von $1$ hat.
\end{flushleft}
