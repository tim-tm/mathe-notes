\chapter{Binomialverteilungen}

\section{Definition}
\begin{flushleft}
    Eine Binomialverteilung ist eine diskrete Wahrscheinlichkeitsverteilung. Das bedeutet, dass eine Binomialverteilung
    auf einer kleineren Menge definiert ist. Oft ist diese Menge endlich (${0,1,2,\dots,n}$), sie kann aber auch abzählbar
    unendlich (Bsp.: Menge der natürlichen Zahlen) sein. \\
    Außerdem kennt eine Binomialverteilung nur zwei Versuchsausgänge, Erfolg oder Misserfolg. Solche Experimente werden Bernoulli
    Experimente genannt.
\end{flushleft}

\section{Aufgabentypen}
\begin{flushleft}
    Für alle hier beschriebenen Aufgabentypen gilt $n$ als die Anzahl aller Versuchsausgänge, $k$ als die beschränkte Anzahl von
    Versuchsausgängen und $p$ als Erfolgswahrscheinlichkeit. \\
    \textit{Beispiel: Ein Würfel wird 100 mal geworfen, wie hoch ist die Wahrscheinlichkeit, dass 5 Sechsen geworfen werden?} \\
    Hier gilt:
    \begin{align}
        n=100,k=5,p=\frac{1}{6}
    \end{align}
\end{flushleft}

\subsection{Bestimmung von $P$}
\begin{flushleft}
    Der häufigste und einfachste Aufgabentyp ist die Bestimmung von $P$.
    Hierbei verlässt man sich auf eine treue Formel.
    \begin{align}
        P(X=k)=\binom{n}{k} \cdot p^k \cdot (1-p)^{n-k}
    \end{align}
    Kumulierte Wahrscheinlichkeiten berechnet man mit der selben Formel, diese benötigen
    jedoch mehr Aufwand.
    \begin{align}
        P(X \leq 2)=P(X=0)+P(X=1)+P(X=2)
    \end{align}
    Mathematica bietet den $CDF$ Befehl um diese Art der kumulierten Wahrscheinlichkeit auszurechnen.
    Somit lässt sich jeder beliebige Ausdruck in solch einen Ausdruck umformen und mit Mathematica berechnen.
    \begin{align}
        P(X>k) &= 1-P(X \leq k) \\
        P(X \geq k) &= 1-P(X \leq k-1) \\
        P(X<k) &= P(X \leq k-1) \\
        P(k_{1} \leq X \leq k_{2}) &= P(X \leq k_{2})-P(X \leq k_{1}-1)
    \end{align}
\end{flushleft}

\subsection{Bestimmung von $n$}
\begin{flushleft}
    In manchen Aufgaben sind nur $p$ und $k$ gegeben, $n$ muss bestimmt werden.
    Aus der Aufgabenstellung geht dann eine ähnliche Ungleichung hervor (hier: $p=0.6$ und $k=n$):
    \begin{align}
        P(X=n) &\leq 0.01 \\
        \binom{n}{n} \cdot 0.6^n \cdot 0.4^0 &\leq 0.01 \\
        0.6^n &\leq 0.01 \\
        \ln(0.6^n) &\leq \ln(0.01) \\
        n\ln(0.6) &\leq \ln(0.01) \\
        n &\geq \frac{\ln(0.01)}{\ln(0.6)} \approx 9.01 \\
        \geq &\text{, da} \ln(0.6)<0
    \end{align}
    Hier ist also, wie so oft in der Stochastik, die einzige Kunst das Entschlüsseln der Aufgabenstellung.
\end{flushleft}

\subsection{Erwartungswert und Standardabweichung}
\begin{flushleft}
    Der Erwartungswert $\mu$ und die Standardabweichung $\sigma$ sind wichtige Bestandteile der Stochastik.
    Glücklicherweise ist es bei Binomialverteilungen relativ einfach $\mu$ und $\sigma$ zu bestimmen.
    \begin{align}
        \mu &= n \cdot p \\
        \sigma &= \sqrt{n \cdot p \cdot (1-p)}
    \end{align}
    Der Erwartungswert ist der Wert, der im Schnitt erwartet wird, wenn ein Experiment oft ausgeführt wird.
    Die Standardabweichung ist die mögliche Abweichung vom Erwartungswert.
    Dieses Interval beschreibt diese Abweichung:
    \begin{align}
        [\mu-\sigma;\mu+\sigma]
    \end{align}
\end{flushleft}
