\chapter{Analytische Geometrie}

\section{Vektorrechnung}
\begin{flushleft}
Ein Vektor kann als die Verschiebung eines Punktes gesehen werden.
Dabei kann jedoch ein einzelner Vektor auf eine beliebige Menge von Punkten angewandt werden.
Wenn man den Vektor $\vec{a}$ auf den Punkt $P$ anwendet, bekommt man also einen zweiten Punkt $P'$,
dessen Koordinaten um die Werte des Vektors $\vec{a}$ verschoben wurden.

\begin{align}
    \vec{a} &= \begin{pmatrix} 1 \\ 1 \\ 1 \end{pmatrix} \\
    P &= \left(1,2,3\right) \\
    P' &= \left(1+1,2+1,3+1\right) = \left(2,3,4\right)
\end{align}
\end{flushleft}

\begin{center}
\begin{tikzpicture}
\begin{axis}[
    view={35}{15},
    axis lines=center,
    width=15cm,height=15cm,
    xtick={1,2,3,4,5},ytick={1,2,3,4,5},ztick={1,2,3,4,5},
    xmin=0,xmax=5,ymin=0,ymax=5,zmin=0,zmax=5,
    xlabel={$x$},ylabel={$y$},zlabel={$z$}
]

\addplot3 [only marks] coordinates {(1,2,3) (2,3,4)};
\addplot3 [no marks,densely dashed] coordinates {(1,0,0) (1,2,0) (1,2,3) (1,2,0) (0,2,0)};
\addplot3 [no marks,densely dashed] coordinates {(2,0,0) (2,3,0) (2,3,4) (2,3,0) (0,3,0)};

\node [above left] at (axis cs:1,2,3) {$P(1,2,3)$};
\node [above right] at (axis cs:2,3,4) {$P'(2,3,4)$};
\draw [->,thick] (axis cs:1,2,3) to (axis cs:1.97,2.97,3.97);
\node [above left] at (axis cs:1.5,2.5,3.5) {$\vec{a}$};

\end{axis}
\end{tikzpicture}
\end{center}

\subsection{Ortsvektoren}
\begin{flushleft}   
Ortsvektoren verschieben nur den Ursprung des Koordinatensystems.
Der Ortsvektor $\overrightarrow{OP}$ hat also die gleichen Koordinaten, wie der Punkt $P$,
da $O=(0,0,0)$ ist.

\begin{align}
    O &= (0,0,0) \\
    P &= (1,2,3) \\
    \overrightarrow{OP} &= \begin{pmatrix} 1 \\ 2 \\ 3 \end{pmatrix}
\end{align}
\end{flushleft}

\begin{center}
\begin{tikzpicture}
\begin{axis}[
    view={35}{15},
    axis lines=center,
    width=15cm,height=15cm,
    xtick={1,2,3,4,5},ytick={1,2,3,4,5},ztick={1,2,3,4,5},
    xmin=-2,xmax=5,ymin=-2,ymax=5,zmin=-2,zmax=5,
    xlabel={$x$},ylabel={$y$},zlabel={$z$}
]

\addplot3 [only marks] coordinates {(0,0,0) (1,2,3)};
\addplot3 [no marks,densely dashed] coordinates {(1,0,0) (1,2,0) (1,2,3)};
\addplot3 [no marks,densely dashed] coordinates {(0,2,0) (1,2,0)};

\node [above right] at (axis cs:1,2,3) {$P(1,2,3)$};
\node [left] at (axis cs:0,0,0) {$O(0,0,0)$};
\draw [->,thick] (axis cs:0,0,0) to (axis cs:0.95,1.95,2.95);
\node [below] at (axis cs:0.3,1.3,2.3) {$\overrightarrow{OP}$};

\end{axis}
\end{tikzpicture}
\end{center}
