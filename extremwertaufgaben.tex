\chapter{Extremwertaufgaben}

\section{Definition}
\begin{flushleft}   
    Bei Extemwertaufgaben geht es darum die maximal/minimal möglichen Flächeninhalte herauszufinden.
    Dabei kann es oft auch um das maximale/minimale Volumen eines Körpers gehen.
    Jedoch maximiert/minimiert sich das Volumen eines Körpers wenn der Flächeninhalt des Mantels maximal/minimal wird.
    Die Aufgabenstellungen zu diesem Aufgabentyp sind meistens sehr spezifisch und brauchen eine Skizze.
\end{flushleft}

\section{Beispiel}
\begin{flushleft}
    In diesem Beispiel soll der maximale Flächeninhalt eines Dreiecks gefunden werden.
    Das Dreieck ist unterhalb der Funktion $f$.
    Die drei Ecken des Dreiecks sind die Punkte $P(u|v)$, $Q(0|0)$ und $R(u|0)$.
    \begin{align}
        f(x)=x^3-6x^2+9x
    \end{align}
\end{flushleft}

\begin{center}
\begin{tikzpicture}   
\begin{axis}[
    axis lines=middle,
    samples=100,
    domain=-1.5:5,
    ymax=7,
    ymin=-5,
    every axis plot/.append style={thick},
]
\addplot[
    color=red,
    name path=A
]
{((\x)*(\x)*(\x))-(6*(\x)*(\x))+(9*(\x))};
\addlegendentry{$f(x)$}

\addplot[draw=none,name path=B] {(3.4/1.5)*(\x)};
\addplot[draw=none,name path=C] {0};
\addplot[red,opacity=0.3] fill between[of=B and C,soft clip={domain=0:1.5}];

\node[label={150:{$Q(0|0)$}},circle,fill,inner sep=1pt] at (axis cs:0,0) {};
\node[label={70:{$P(u|v)$}},circle,fill,inner sep=1pt] at (axis cs:1.5,3.4) {};
\node[label={60:{$R(u|0)$}},circle,fill,inner sep=1pt] at (axis cs:1.5,0) {};

\end{axis}
\end{tikzpicture}
\end{center}

\begin{flushleft}
    Um die Fläche $A$ eines Dreiecks auszurechnen nutzt man diese Formel, $g$ steht für die Grundseite und $h$ für die Höhe:
    \begin{align}
        A=\frac{gh}{2}
    \end{align}
    Mit $g$ und $h$ sieht der Plot so aus:
\end{flushleft}

\begin{center}
\begin{tikzpicture}   
\begin{axis}[
    axis lines=middle,
    samples=100,
    domain=-1.5:4,
    ymax=7,
    ymin=-5,
    every axis plot/.append style={thick},
]
\addplot[
    color=red,
    name path=A
]
{((\x)*(\x)*(\x))-(6*(\x)*(\x))+(9*(\x))};
\addlegendentry{$f(x)$}

\addplot[draw=none,name path=B] {(3.4/1.5)*(\x)};
\addplot[draw=none,name path=C] {0};
\addplot[red,opacity=0.3] fill between[of=B and C,soft clip={domain=0:1.5}];

\node[label={150:{$Q(0|0)$}},circle,fill,inner sep=1pt] at (axis cs:0,0) {};
\node[label={70:{$P(u|v)$}},circle,fill,inner sep=1pt] at (axis cs:1.5,3.4) {};
\node[label={270:{$R(u|0)$}},circle,fill,inner sep=1pt] at (axis cs:1.5,0) {};

\node[label={0:{$h$}},draw=none,inner sep=1pt] at (axis cs:1.5,1.5) {};
\node[label={250:{$g$}},draw=none,inner sep=1pt] at (axis cs:0.75,0) {};

\end{axis}
\end{tikzpicture}
\end{center}

\begin{flushleft}
    In unserem Beispiel ist $g=u$ und $h=v$, dazu ist $v=f(u)$.
    Wenn wir diese Werte in $A$ einsetzen bildet sich diese Funktion:
    \begin{align}
        A(u) &=\frac{u*f(u)}{2} \\
        A(u) &=\frac{u \left(u^3-6u^2+9u\right)}{2} \\
        A(u) &=\frac{u^4-6u^3+9u^2}{2} \\
        A(u) &=\frac{u^4}{2}-3u^3+\frac{9}{2}u^3
    \end{align}
    Unsere neue Funktion $A(u)$ kann jetzt den Flächeninhalt für jeden Wert von $u$ berechnen.
    Wir brauchen den maximalen Flächeninhalt, also soll die Funktion für den Flächeninhalt ($A(u)$) maximal werden.
    Um ein maximum zu berechnen bilden wir die ersten zwei Ableitungen unserer Funktion.
    \begin{align}
        A(u) &=\frac{u^4}{2}-3u^3+\frac{9}{2}u^3 \\
        A'(u) &= 2u^3-9u^2+9u \\
        A''(u) &= 6u^2-18u+9
    \end{align}
    Jetzt finden wir die Nullstelle der ersten Ableitung.
    \begin{align}
        A'(u) &=0 \\
        \iff u_1=0, u_2 &=\frac{3}{2}, u_3=3
    \end{align}
    Danach prüfen wir, in welcher Stelle ein Maximum liegt.
    \begin{align}
        A''\left(0\right) = 9 & > 0 \Longrightarrow \text{Minimum} \\
        A''\left(\frac{3}{2}\right) = \frac{-9}{2} & < 0 \Longrightarrow \text{Maximum} \\
        A''\left(3\right) = 9 & > 0 \Longrightarrow \text{Minimum} \\
        A\left(\frac{3}{2}\right) &= \frac{81}{32} \\
        v=f\left(\frac{3}{2}\right) &= \frac{27}{8}
    \end{align}
    Der größte Flächeninhalt ist $\frac{81}{32}$, um diesen zu erreichen muss $u=\frac{3}{2}$ und $v=\frac{27}{8}$ sein.
\end{flushleft}

\begin{center}
\begin{tikzpicture}   
\begin{axis}[
    axis lines=middle,
    samples=100,
    domain=-1.5:4,
    ymax=7,
    ymin=-5,
    every axis plot/.append style={thick},
]
\addplot[
    color=red,
    name path=A
]
{((\x)*(\x)*(\x))-(6*(\x)*(\x))+(9*(\x))};
\addlegendentry{$f(x)$}

\addplot[draw=none,name path=B] {((27/8)/(3/2))*(\x)};
\addplot[draw=none,name path=C] {0};
\addplot[red,opacity=0.3] fill between[of=B and C,soft clip={domain=0:1.5}];

\addplot[color=black,very thick] coordinates {(1.5,0) (1.5,3.375)};
\addplot[color=black,very thick] coordinates {(0,0) (1.5,0)};

\node[label={0:{$\frac{27}{8}$}},draw=none,inner sep=1pt] at (axis cs:1.5,1.5) {};
\node[label={250:{$\frac{3}{2}$}},draw=none,inner sep=1pt] at (axis cs:0.75,0) {};

\end{axis}
\end{tikzpicture}
\end{center}
